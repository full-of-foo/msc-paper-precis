 %----------------------------------------------------------------------------------------
%	PACKAGES AND OTHER DOCUMENT CONFIGURATIONS
%----------------------------------------------------------------------------------------

\documentclass[twoside]{article}

\usepackage{lipsum} % Package to generate dummy text throughout this template

\usepackage[sc]{mathpazo} % Use the Palatino font
\usepackage[T1]{fontenc} % Use 8-bit encoding that has 256 glyphs
\linespread{1.05} % Line spacing - Palatino needs more space between lines
\usepackage{microtype} % Slightly tweak font spacing for aesthetics

\usepackage[hmarginratio=1:1,top=32mm,columnsep=20pt]{geometry} % Document margins
\usepackage{multicol} % Used for the two-column layout of the document
\usepackage[hang, small,labelfont=bf,up,textfont=it,up]{caption} % Custom captions under/above floats in tables or figures
\usepackage{booktabs} % Horizontal rules in tables
\usepackage{float} % Required for tables and figures in the multi-column environment - they need to be placed in specific locations with the [H] (e.g. \begin{table}[H])
\usepackage{hyperref} % For hyperlinks in the PDF

\usepackage{lettrine} % The lettrine is the first enlarged letter at the beginning of the text
\usepackage{paralist} % Used for the compactitem environment which makes bullet points with less space between them

\usepackage{abstract} % Allows abstract customization
\renewcommand{\abstractnamefont}{\normalfont\bfseries} % Set the "Abstract" text to bold
\renewcommand{\abstracttextfont}{\normalfont\small\itshape} % Set the abstract itself to small italic text

\usepackage{titlesec} % Allows customization of titles
\renewcommand\thesection{\Roman{section}} % Roman numerals for the sections
\renewcommand\thesubsection{\Roman{subsection}} % Roman numerals for subsections
\titleformat{\section}[block]{\large\scshape\centering}{\thesection.}{1em}{} % Change the look of the section titles
\titleformat{\subsection}[block]{\large}{\thesubsection.}{1em}{} % Change the look of the section titles

\usepackage{fancyhdr} % Headers and footers
\pagestyle{fancy} % All pages have headers and footers
\fancyhead{} % Blank out the default header
\fancyfoot{} % Blank out the default footer
\fancyhead[C]{CA640 $\bullet$ 21/Nov/14 $\bullet$ Pr�cis Assignment} % Custom header text
\fancyfoot[RO,LE]{\thepage} % Custom footer text

\usepackage{natbib}
\bibliographystyle{agsm}

%----------------------------------------------------------------------------------------
%	TITLE SECTION
%----------------------------------------------------------------------------------------

\title{\vspace{-15mm}\fontsize{24pt}{10pt}\selectfont\textbf{Paper Pr�cis: Creating a Shared Understanding of Testing Culture on a Social Coding Site}} % Article title

\author{
\large
\textsc{Anthony Troy }\\[2mm] 
\normalsize \href{mailto:anthony.troy3@mail.dcu.ie}{anthony.troy3@mail.dcu.ie | 14212116} \\
\vspace{-5mm}
}
\date{}

%----------------------------------------------------------------------------------------

\begin{document}

\maketitle % Insert title

\thispagestyle{fancy} % All pages have headers and footers

%----------------------------------------------------------------------------------------
%	Disclaimer, Pr�cis & ABSTRACT
%----------------------------------------------------------------------------------------
\renewcommand{\abstractname}{\vspace{-\baselineskip}}
\begin{abstract}
\noindent \textbf{Disclaimer}: Submitted to Dublin City University, School of Computing for module CA640: Research Skills, 2014/2015. I hereby certify that the work presented and the material contained herein is my own except where explicitly stated references to other material are made.
\end{abstract}

\vspace{-7mm}

\renewcommand{\abstractname}{\vspace{-\baselineskip}}
\begin{abstract}
\noindent  \textbf{Basing Paper}: R. Pham, L. Singer, O. Liskin, F. Figueira Filho, and K. Schneider. Creating a shared understanding of testing culture on a social coding site. In Proceedings of the 2013 International Conference on Software Engineering, ICSE 2013, pages 112-121, Piscataway, NJ, USA, 2013. IEEE Press.
\end{abstract}

\vspace{1mm}
\renewcommand{\abstractname}{Abstract}
\begin{abstract}
\noindent 
Software engineering is inherently collaborative, at the team, project and community levels. However many project
owners battle with establishing and communicating a testing culture that is appropriate for their project's needs.
Transparent software environments such as Github incorporate social media features to make work more visible. 
Previous research forwards that this transparency influences the behaviour of developers. Our basing paper
extends upon those findings to investigate how the increased transparency found on GitHub influences testing 
behaviours.\par
Derived from interviewing active GitHub users, both project owners and developers, our basing paper 
suggests several strategies that software teams can use to positively influence the testing behaviour in their projects. 
Additionally, it is found that project owners on GitHub are often not aware of these strategies. This research reports
on the difficulties and risks which this causes and suggests guidelines for developing a sustainable testing culture.

\end{abstract}

%----------------------------------------------------------------------------------------
%	ARTICLE CONTENTS
%----------------------------------------------------------------------------------------

\begin{multicols}{2} % Two-column layout throughout the main article text

\section{Introduction}

Like many fields that rely closely on collaboration
and coordination, Software Engineering is rapidly transforming. The emergence of
social media sites --- both general and those specifically created for software 
developers --- has resulted in a paradigm shift in the field; developers 
can now connect with, provide help to, collaborate with, and learn from one another 
with ease \citep{Storey}. \par
By virtue social media sites provide a high degree of social transparency, 
making freely available the "social meta-data" around information exchange 
\citep{Stuart}. \citet{Dabbish} forwards that the transparency GitHub provides
influences the behaviour of software developers. Our basing paper investigates 
how testing behaviour is influenced by such social coding sites. \citet{Pham},
the paper author, poses that there exists no previous work on this topic and that
"if social coding sites can influence the testing behaviour of developers, they
might also have an impact on the progress of software development projects 
and their resulting projects". \par 
\setlength{\parskip}{0em} % hack 
In accordance with our basing paper, this 
pr�cis is structured as follows. In section II, we describe some related works
around this topic, furthering from this section III outlines the respective study 
undertaken. Section IV discusses the emerged findings and section V concludes 
this pr�cis.

%------------------------------------------------

\section{Related Work}

Like many fields that rely closely on collaboration
and coordination, Software Engineering is rapidly transforming. The emergence of
social media sites --- both general and those specifically created for software 
developers --- has resulted in a paradigm shift in the field; developers 
can now connect with, provide help to, collaborate with, and learn from one another 
with ease \citep{Storey}. \par
By virtue social media sites provide a high degree of social transparency, 
making freely available the "social meta-data" around information exchange 
\citep{Stuart}. \citet{Dabbish} forwards that the transparency GitHub provides
influences the behaviour of software developers. Our basing paper investigates 
how testing behaviour is influenced by such social coding sites. \citet{Pham},
the paper author, poses that there exists no previous work on this topic and that
"if social coding sites can influence the testing behaviour of developers, they
might also have an impact on the progress of software development projects 
and their resulting projects". \par 
\setlength{\parskip}{0em} % hack 
In accordance with our basing paper, this 
pr�cis is structured as follows. In section II, we describe some related works
around this topic, furthering from this section III outlines the respective study 
undertaken. Section IV discusses the emerged findings and section V concludes 
this pr�cis.

%------------------------------------------------

\section{Study Design}
\lipsum[2] % Dummy text

%------------------------------------------------

\section{Findings}
\lipsum[2] % Dummy text

%------------------------------------------------

\section{Discussion}
\subsection{Subsection One}
Maecenas sed ultricies felis. Sed imperdiet dictum arcu a egestas. 
\begin{compactitem}
\item Donec dolor arcu, rutrum id molestie in, viverra sed diam
\item Curabitur feugiat
\item turpis sed auctor facilisis
\item arcu eros accumsan lorem, at posuere mi diam sit amet tortor
\item Fusce fermentum, mi sit amet euismod rutrum
\item sem lorem molestie diam, iaculis aliquet sapien tortor non nisi
\item Pellentesque bibendum pretium aliquet
\end{compactitem}
\lipsum[1] % Dummy text
\subsection{Subsection Two}
\lipsum[1] % Dummy text

%------------------------------------------------

\section{Discussion}
\lipsum[2] % Dummy text

%----------------------------------------------------------------------------------------
%	REFERENCE LIST
%----------------------------------------------------------------------------------------

\bibliography{article_2}

%----------------------------------------------------------------------------------------

\end{multicols}

\end{document}
